% Options for packages loaded elsewhere
\PassOptionsToPackage{unicode}{hyperref}
\PassOptionsToPackage{hyphens}{url}
%
\documentclass[
]{article}
\usepackage{lmodern}
\usepackage{amssymb,amsmath}
\usepackage{ifxetex,ifluatex}
\ifnum 0\ifxetex 1\fi\ifluatex 1\fi=0 % if pdftex
  \usepackage[T1]{fontenc}
  \usepackage[utf8]{inputenc}
  \usepackage{textcomp} % provide euro and other symbols
\else % if luatex or xetex
  \usepackage{unicode-math}
  \defaultfontfeatures{Scale=MatchLowercase}
  \defaultfontfeatures[\rmfamily]{Ligatures=TeX,Scale=1}
\fi
% Use upquote if available, for straight quotes in verbatim environments
\IfFileExists{upquote.sty}{\usepackage{upquote}}{}
\IfFileExists{microtype.sty}{% use microtype if available
  \usepackage[]{microtype}
  \UseMicrotypeSet[protrusion]{basicmath} % disable protrusion for tt fonts
}{}
\makeatletter
\@ifundefined{KOMAClassName}{% if non-KOMA class
  \IfFileExists{parskip.sty}{%
    \usepackage{parskip}
  }{% else
    \setlength{\parindent}{0pt}
    \setlength{\parskip}{6pt plus 2pt minus 1pt}}
}{% if KOMA class
  \KOMAoptions{parskip=half}}
\makeatother
\usepackage{xcolor}
\IfFileExists{xurl.sty}{\usepackage{xurl}}{} % add URL line breaks if available
\IfFileExists{bookmark.sty}{\usepackage{bookmark}}{\usepackage{hyperref}}
\hypersetup{
  hidelinks,
  pdfcreator={LaTeX via pandoc}}
\urlstyle{same} % disable monospaced font for URLs
\usepackage[margin=1in]{geometry}
\usepackage{longtable,booktabs}
% Correct order of tables after \paragraph or \subparagraph
\usepackage{etoolbox}
\makeatletter
\patchcmd\longtable{\par}{\if@noskipsec\mbox{}\fi\par}{}{}
\makeatother
% Allow footnotes in longtable head/foot
\IfFileExists{footnotehyper.sty}{\usepackage{footnotehyper}}{\usepackage{footnote}}
\makesavenoteenv{longtable}
\usepackage{graphicx,grffile}
\makeatletter
\def\maxwidth{\ifdim\Gin@nat@width>\linewidth\linewidth\else\Gin@nat@width\fi}
\def\maxheight{\ifdim\Gin@nat@height>\textheight\textheight\else\Gin@nat@height\fi}
\makeatother
% Scale images if necessary, so that they will not overflow the page
% margins by default, and it is still possible to overwrite the defaults
% using explicit options in \includegraphics[width, height, ...]{}
\setkeys{Gin}{width=\maxwidth,height=\maxheight,keepaspectratio}
% Set default figure placement to htbp
\makeatletter
\def\fps@figure{htbp}
\makeatother
\setlength{\emergencystretch}{3em} % prevent overfull lines
\providecommand{\tightlist}{%
  \setlength{\itemsep}{0pt}\setlength{\parskip}{0pt}}
\setcounter{secnumdepth}{5}

\author{}
\date{\vspace{-2.5em}}

\begin{document}

{
\setcounter{tocdepth}{2}
\tableofcontents
}
\hypertarget{introduction}{%
\section{Introduction}\label{introduction}}

Cognitive impairment (CI) constitutes one of the most common manifestations of the neuropsychiatric syndromes experienced by patients living with Systemic lupus erythematosus (SLE), with a prevalence of 38\% (95\% confidence interval: 33-43\%).\textsuperscript{1} Conventional screening/diagnostic tools for CI in SLE patients take into account the scores generated by neuropsychological batteries like the American College of Rheumatology Neuropsychological Battery (ACR-NB).

Given the complexity of diagnosing CI in SLE patients, there is an unmet need to identify the best screening, diagnostic metrics of CI and furthermore the assessment of cognitive function over time. Therefore we propose a new framework based on Hidden Markov Models (HMMs) to address the aforementioned challenges for classifing SLE patients into CI vs non CI-groups from patients followed at the Toronto Lupus Clinic-Cognitive Study.

\hypertarget{methods}{%
\section{Methods}\label{methods}}

The study population consisted of 301 consecutive SLE patients followed at the Toronto Lupus Clinic-Cognitive Study, who consented to participate and to be assessed for CI. We have already assessed all 301 SLE patients at baseline using the comprehensive ACR-NB (187 patients also completed visits at 6 months and at 12 months).

This battery evaluates the major cognitive domains: Manual motor speed and dexterity, simple attention and processing speed, visual-spatial construction, verbal fluency, learning and memory, executive functioning and assesses estimated pre-morbid IQ. Patient scores are compared to a normative sample of age- and gender-matched healthy controls to obtain z-scores. CI was operationalized on the battery as a z-score of ≤-1.5 (as compared to controls) on ≥2 domains or z ≤-2.0 on ≥1 domain.

Instead of using the binary operalization of CI, this new approach consists of first step reducing the high-dimensional aspect of the ACR-NB using principal component analysis (PCA) with the objective to create a single component score which explains the most variance (i.e.~first component) . Afterwards this approach builds a 2-state CI discrete-time HMM with the dimensionality reduction gained in the first step. The HMM proposed in this study assumes that the progression of CI in patients with SLE can be segmented into 2 distinct disease states, where each disease state captures if a patient is impaired or not at time \(t\), using the observed data from our PCA. Additionally we adjusted our resulting component score by education level, meaning that the hidden states captured the unobserved heterogeneity not explained after adjusting by this covariate. All the statistical analysis was done from a Bayesian perspective.

\hypertarget{results}{%
\section{Results}\label{results}}

\ref{fig:fig0} rerer

\begin{figure}
\centering
\includegraphics{Abstract_files/figure-latex/fig0-1.pdf}
\caption{\label{fig:fig0}Biplot at baseline}
\end{figure}

sdds

\begin{figure}
\centering
\includegraphics{Abstract_files/figure-latex/fig1-1.pdf}
\caption{\label{fig:fig1}Posterior densities}
\end{figure}

\hypertarget{references}{%
\section*{References}\label{references}}
\addcontentsline{toc}{section}{References}

\hypertarget{refs}{}
\leavevmode\hypertarget{ref-al2018prevalence}{}%
1. Al Rayes, H. \emph{et al.} What is the prevalence of cognitive impairment in lupus and which instruments are used to measure it? A systematic review and meta-analysis. in \emph{Seminars in arthritis and rheumatism} \textbf{48}, 240--255 (Elsevier, 2018).

\end{document}
